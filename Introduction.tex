\chapter*{Introduction générale}
\markboth {Introduction générale}{}
\addcontentsline{toc}{chapter}{Introduction générale}


Dans le contexte actuel de la gestion d’entreprise axée sur l’efficacité opérationnelle et la rentabilité, l’optimisation des ressources et des actifs est devenue un impératif stratégique pour les organisations modernes. En effet, dans un environnement économique de plus en plus concurrentiel, les entreprises doivent non seulement maximiser l’utilisation de leurs ressources, mais également minimiser les coûts liés à leur maintenance et leur exploitation. Parmi ces actifs essentiels, les parcs d’entreprise, qu’ils comprennent des véhicules, des flottes de transport ou d’autres équipements, représentent des investissements substantiels. Ces investissements nécessitent une gestion rigoureuse et proactive pour assurer leur durabilité, leur efficacité et leur rentabilité à long terme.\\

Face à cette exigence croissante, notre projet s’attache à répondre aux besoins de gestion des parcs d’entreprise à travers le développement d’une application mobile novatrice et fonctionnelle, intégrant les principes de la GMAO (Gestion de Maintenance Assistée par Ordinateur). Cette solution numérique vise à rationaliser les processus de gestion en offrant aux parties prenantes une plateforme intégrée pour surveiller, entretenir et optimiser les parcs d’entreprise de manière efficace et transparente. L’objectif est de fournir un outil qui non seulement facilite la gestion quotidienne, mais aussi anticipe les besoins futurs grâce à des fonctionnalités avancées de suivi et d’analyse des données.\\

Dans le cadre de ce rapport, nous présentons une analyse approfondie du développement de cette application, mettant en lumière nos objectifs stratégiques, les fonctionnalités clés de l’application, ainsi que les défis rencontrés et les stratégies d’atténuation mises en œuvre. Nous détaillerons comment notre solution permet une gestion proactive grâce à des notifications automatisées pour les maintenances, des rapports détaillés sur l’utilisation et l’état des actifs, et des tableaux de bord personnalisables pour une vision claire et en temps réel des performances.\\

Nous explorerons également les implications potentielles de cette solution pour les entreprises utilisatrices, notamment en termes d’amélioration de l’efficacité opérationnelle, de réduction des coûts et d’optimisation des performances. En réduisant les temps d’arrêt des équipements et en prolongeant leur durée de vie, notre application vise à générer des économies significatives et à améliorer la productivité globale des parcs d’entreprise. De plus, en fournissant des données précises et exploitables, elle permet aux gestionnaires de prendre des décisions éclairées, d’optimiser les processus et de mieux planifier les investissements futurs.\\

Enfin, ce rapport abordera les perspectives d’évolution de notre application, avec des plans pour intégrer des améliorations continues basées sur les retours d'expérience des utilisateurs et les besoins émergents des entreprises. Notre vision est de créer une solution qui évolue avec les besoins des entreprises, offrant une flexibilité et une scalabilité pour répondre aux défis de demain.